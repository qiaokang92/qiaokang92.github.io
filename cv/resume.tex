%%%%%%%%%%%%%%%%%%%%%%%%%%%%%%%%%%%%%%%%%
% Medium Length Professional CV
% LaTeX Template
% Version 2.0 (8/5/13)
%
% This template has been downloaded from:
% http://www.LaTeXTemplates.com
%
% Original author:
% Rishi Shah
%
% Important note:
% This template requires the resume.cls file to be in the same directory as the
% .tex file. The resume.cls file provides the resume style used for structuring the
% document.
%
%%%%%%%%%%%%%%%%%%%%%%%%%%%%%%%%%%%%%%%%%

%----------------------------------------------------------------------------------------
%	PACKAGES AND OTHER DOCUMENT CONFIGURATIONS
%----------------------------------------------------------------------------------------

\documentclass{resume} % Use the custom resume.cls style

\usepackage[left=0.75in,top=0.6in,right=0.75in,bottom=0.6in]{geometry} % Document margins
\newcommand{\tab}[1]{\hspace{.2667\textwidth}\rlap{#1}}
\newcommand{\itab}[1]{\hspace{0em}\rlap{#1}}
\name{Qiao Kang} % Your name
\address{PhD student, Rice University, Houston, TX 77005} % Your address
\address{(+1) 346-317-2056,  qiaokang1213@gmail.com, wechat: buaakq}
\address{https://qiaokang.org}

\begin{document}


%\begin{rSection}{Research interests}
%I am broadly  interested in networking, distributed systems, and security.
%My current research focus is leveraging programmable network devices
%for efficient and secure networked systems.
%\end{rSection}


I will leave my PhD program with an MS degree on 12/30/2020.
I'm looking for a software engineer job for system software development,
such as OS kernels, networking and distributed systems.


\begin{rSection}{Education}

{\bf Rice University, USA} \hfill { Aug 2018 - Dec 2020}
\\ Ph.D. student (will leave and graduate with an MS degree)

{\bf Beihang University, China} \hfill { Sep 2014 - Jun 2017}
\\ MS in Software Engineering

{\bf Beihang University, China} \hfill { Sep 2010 - Jun 2014}
\\ BS in Software Engineering

\end{rSection}



\begin{rSection}{Work Experience}

\item {\bf OS kernel developer, VMware} \hfill { Jan 2017 - Jun 2018}

My main job was Network Interface Card (NIC) driver development in the
ESXi kernel.
I developed ESXi NIC driver for the Virtio virtualization framework from
scratch.
I also Maintained the Solarflare 10G/40G NIC drivers.

\vspace{2mm}

\end{rSection}


\begin{rSection}{Research Experience}

\vspace{2mm}

At Rice University, my research work centers around emerging network hardware,
such as P4-programmable switches and SmartNICs.
My first two projects (Poise and NetWarden) seek to leverage P4 switches to build more secure networks.
My third project (P4wn) contrinbutes a novel profiling tool for stateful P4 programs.
My fourth project (Clara) aims at providing performance prediction for SmartNIC offloading.

\vspace{2mm}

\item {\bf Clara: Performance Clarity for SmartNIC Offloading} \hfill { April 2020 - Present}

Offloading packet processing programs from CPUs to SmartNICs can bring significant performance benefits,
but the developer has no easy way to understand the offloaded performance beforehand.
We develop Clara, an automated tool to analyze a legacy NF in its unported form,
identify acceleration opportunities, and predict its offloaded performance.
Our experiments using Click NFs and a Netronome SmartNIC demonstrate that Clara
can provide useful offloading hints and reasonable prediction accuracy.
({\bf HotNets'20})
%We have published a workshop paper in the {\bf HotNets'20 Workshop}.
%Our full paper is in progress.

\vspace{3mm}

\item {\bf P4wn: Probabilistic Profiling of Stateful Data Planes (project lead)}    \hfill { May 2019 - Present}

There is a flurry of projects that develop networked systems in P4-programmable switches,
but existing program profiling tools are unable to catch up.
We develop P4wn, a program profiler that can analyze stateful program behaviors
of recent P4 systems, as well as the behavior probabilities, in a scalable manner.
We show P4wn is useful to discovering adversarial inputs to these P4 systems
by distinguishing and stressing the program ``edge cases''.
%We have published a preliminary paper in the {\bf USENIX CSET'19 Workshop}
%and our full conference paper is under review.
({\bf CSET'19, ASPLOS'21})

\vspace{3mm}

\item {\bf NetWarden: Mitigating Network Covert Channels without Performance Loss} \hfill { May 2019 - Nov 2019}

Network covert channels are an advanced class of attacks to modern networks.
Traditional solutions rely on general-purpose CPUs to detect and mitigate them,
but they can incur high performance overhead.
We propose NetWarden, which leverages P4 switches to mitigate network covert channels
at switch data planes.
Our experiments show that NetWarden can achieve similar detection accuracy and mitigation
effectiveness compared with existing solutions, but without hurting end-to-end TCP performance.
%Our research paper is published at {\bf USENIX Security'20}, a top-tier conference in Computer Security.
({\bf USENIX Security'20})

\vspace{3mm}

\item {\bf Poise: Programmable In-Network Security for BYOD Policies (project lead)} \hfill { Aug 2018 - Sep 2019}

We present a new security paradigm called programmable in-network security (Poise) for
enforcing access control policies in enterprise networks.
Administrators write security policies using an easy-to-use, high-level language,
and our Poise compiler can generate P4 programs, which will be installed in programmable switches,
to enforce these policies.
Poise outperforms traditional solutions which are based on OpenFlow networks, and
is resilient to control plane saturation attacks.
%Our research paper is published at {\bf USENIX Security'20}, a top-tier conference in Computer Security.
({\bf CSET'19, USENIX Security'20})

\vspace{2mm}

\end{rSection}

\begin{rSection}{Research publications}


\item {\bf Probabilistic Profiling of Stateful Data Planes for Adversarial Testing}\\
{\bf Qiao Kang}, Jiarong Xing, Yiming Qiu, and Ang Chen \\
26th International Conference on Architectural Support for Programming Languages and Operating Systems ({\bf ASPLOS'21}), Virtual, April 2021

\vspace{3mm}

\item {\bf Clara: Performance Clarity for SmartNIC Offloading}\\
Yiming Qiu$^\star$, {\bf Qiao Kang$^\star$}, Ming Liu, and Ang Chen \\
19th ACM Workshop on Hot Topics in Networks ({\bf HotNets'20}), Virtual, Nov 2020

\vspace{3mm}

\item {\bf Mitigating Network Covert Channels while Preserving Performance}\\
Jiarong Xing, {\bf Qiao Kang}, and Ang Chen\\
29th USENIX Security Symposium ({\bf Security'20}), Virtual, Aug 2020

\vspace{3mm}

\item {\bf Programmable In-Network Security for Context-aware BYOD Policies}\\
{\bf Qiao Kang$^\star$}, Lei Xue$^\star$, Adam Morrison$^\star$, Yuxin Tang, Ang Chen, and Xiapu Luo\\
29th USENIX Security Symposium ({\bf Security'20}), Virtual, Aug 2020

\vspace{3mm}

\item {\bf Automated Attack Discovery in Data Plane Systems}\\
{\bf Qiao Kang}, Jiarong Xing and Ang Chen\\
12th USENIX Workshop on Cyber Security Experimentation and Test ({\bf CSET'19}), Santa Clara, CA, USA, Aug 2019

\vspace{1mm}

($^\star$ indicates equal contribution)

\vspace{2mm}

\end{rSection}



\begin{rSection}{Talks}

\item Qiao Kang, ``Programmable In-Network Security for Context-aware BYOD Policies'' \\
USENIX Security'20 (online), Aug 2020

\item Qiao Kang, ``Programmable In-Network Security for Context-aware BYOD Policies'' \\
Computer Systems Lab, University of Washington (online), Jul 2020

\item Qiao Kang, ``Programmable In-Network Security for Context-aware BYOD Policies'' \\
P4 Expert Roundtable Series (online), Apr 2020

\item Qiao Kang, ``Automated Attack Discovery in Data Plane Systems'' \\
CSET'19, Santa Clara, CA, USA, Aug 2019

\end{rSection}


%\begin{rSection}{Talks}
%\item Teaching assistant: Secure and Cloud Computing (COMP 436/536, Fall 2020)
%\end{rSection}


\begin{rSection}{Skills} \itemsep -3pt

%\begin{tabular}{ @{} >{\bfseries}l @{\hspace{6ex}} l }
%Programming \ &
\item Expertise in C programming, familiar with C++, Python and Shell
\item Familiar with OS kernel development, especially NIC drivers
%\item Familiar with recent advances in networking, especially {\em programmable data planes}
\item Familiar with P4 (a domain-specific language for network data plane programming)
%\end{tabular}

\end{rSection}


%----------------------------------------------------------------------------------------
% Extra Curricular
%----------------------------------------------------------------------------------------
%\begin{rSection}{Awards} \itemsep -3pt
%\item Outstanding Graduate Award of Beihang University, 2017
%\item National Scholarship Award of China, 2013
%\item First-prize Scholarship of Academic Performance of Beihang University, 2013
%\end{rSection}


\end{document}
